\section{Statystyki}

W celu prezentacji statystyk przeprowadzone zostały wielokrotne przebiegi symulacji.
Wyniki te pozwolą na określenie średniej wielkości zniszczeń i ekstremów.

Pierwszy zestaw symulacji został przeprowadzony dla parametrów:
\begin{itemize}
\item Wymiary: $200m x 200m$
\item Promień (max): $7m/s$
\item Prędkość trawersalna (max): $12m/s$
\item Prędkość radialna (max): $13m/s$
\item Prędkość translacji: $14m/s$
\item Punkt startowy: $(-8, -8)$
\item Kąt przemieszczania się wiru: $45$ stopni
\item Odstępy między drzewami: $9m$
\item Rozkład drzew: Jednorodny
\end{itemize}

Wyniki przedstawia tabela~\ref{tab:sym1}:

 \begin{table}[h!]
 \caption{Wyniki dla rozkładu jednorodnego.}
 \label{tab:sym1}
 \begin{center}
\begin{tabular} {|c|c|c|}
\hline
Wyrwane & Złamane & Początkowa liczba \\ \hline \hline
$28$ 	& $47$	&	$529$ \\ \hline
$37$ 	& $52$ 	&	$529$ \\ \hline
$34$ 	& $53$ 	&	$529$ \\ \hline
$46$ 	& $47$	&	$529$ \\ \hline
$42$ 	& $60$	&	$529$ \\ \hline
$34$ 	& $47$ 	&	$529$ \\ \hline
\end{tabular}
\end{center}
\end{table}


Analizując tabelę można stwierdzić, że stopień zniszczeń lasu dla zadanych parametrów nie jest duży. Średnia liczba wyrwanych drzew wyniosła około $37$ sztuk, w przypadku złamań: $51$. Największa liczba wyrwanych drzew jaka wystąpiła w przebiegach symulacji to $46$, najmniejsza: $28$. W przypadku złamań jest to odpowiednio $60$ i $47$.
Średnio zniszczonych zostało $16.64\%$ drzew.

Kolejny zestaw symulacji został przeprowadzony dla parametrów:
\begin{itemize}
\item Wymiary: $100m x 100m$
\item Promień (max): $7m/s$
\item Prędkość trawersalna (max): $12m/s$
\item Prędkość radialna (max): $13m/s$
\item Prędkość translacji: $14m/s$
\item Punkt startowy: $(-8, -8)$
\item Kąt przemieszczania się wiru: $45$ stopni
\item Rozkład drzew: Losowy
\end{itemize}

Wyniki przedstawia tabela~\ref{tab:sym2}:

 \begin{table}[h!]
 \caption{Wyniki dla rozkładu losowego.}
 \label{tab:sym2}
 \begin{center}
\begin{tabular} {|c|c|c|}
\hline
Wyrwane & Złamane & Początkowa liczba \\ \hline \hline
$30$ 	& $32$ 	&	$100$ \\ \hline
$23$ 	& $30$ 	&	$100$ \\ \hline
$33$ 	& $30$ 	&	$100$ \\ \hline
$27$ 	& $35$ 	&	$100$ \\ \hline
$24$ 	& $28$ 	&	$100$ \\ \hline
$26$ 	& $25$ 	&	$100$ \\ \hline
\end{tabular}
\end{center}
\end{table}


Stopień zniszczeń jest widocznie większy niż w poprzednich symulacjach. Spowodowane jest to głównie znacząco mniejszą liczbą drzew jak i nierównomiernym zagęszczeniem. Średnia liczba wyrwanych drzew wyniosła $27$, najmniejsza $23$, największa $33$. W przypadku złamanych drzew jest to odpowiednioi $30$, $25$, $35$.
Średnio zniszczonych zostało $57\%$ drzew.

Ostatni zestaw symulacji został przeprowadzony dla parametrów:
\begin{itemize}
\item Wymiary: $100m x 100m$
\item Promień (max): $7m/s$
\item Prędkość trawersalna (max): $12m/s$
\item Prędkość radialna (max): $13m/s$
\item Prędkość translacji: $14m/s$
\item Punkt startowy: $(-8, -8)$
\item Kąt przemieszczania się wiru: $45$ stopni
\item Rozkład drzew: Plama
\end{itemize}

Wyniki przedstawia tabela~\ref{tab:sym3}:

 \begin{table}[h!]
 \caption{Wyniki dla rozkładu typu `plama'.}
 \label{tab:sym3}
 \begin{center}
\begin{tabular} {|c|c|c|}
\hline
Wyrwane & Złamane & Początkowa liczba \\ \hline \hline
$26$ 	& $21$ 	&	$100$ \\ \hline
$14$ 	& $10$ 	&	$100$ \\ \hline
$19$ 	& $50$ 	&	$100$ \\ \hline
$11$ 	& $18$ 	&	$100$ \\ \hline
$24$ 	& $31$ 	&	$100$ \\ \hline
$7$ 	& $12$ 	&	$100$ \\ \hline
\end{tabular}
\end{center}
\end{table}

Wyniki tych przebiegów są bardzo zróżnicowane. Wynika to ze sposobu generowania lasu - niekiedy mogło się zdarzyć, iż wir nie przechodził bezpośrednio przez las, lecz po jego krawędzi, co wpłynęło znacząco na stopień zniszczeń. 
Średnia wyrwanych drzew: $17$, złamanych: $24$. Minimalna liczba wyrwanych to $7$ drzew, maksymalna $26$. Odpowiednio dla złamanych jest to $10$ i $50$.
Średnio zniszczonych zostało $41\%$ drzew.

