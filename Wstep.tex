\section{Wstęp}

Celem projektu jest stworzenie symulacji przemieszczania się tornada przez las oraz analiza zniszczeń drzew. Modelem wykorzystanym w symulacji wiru jest model Rankine (rozdział 3), łamliwość drzew symulowana jest na podstawie modelu HWIND (rozdział 2). W projekcie wykorzystane zostały koncepcje charakterystyczne dla programowania agentowego -- występujące byty działają niezależnie na osobnych wątkach, komunikując się ze sobą asynchronicznie.

Symulacje tego typu nie są bardzo popularne w Polsce ze względu na rzadkość ich występowania. Jednak z powodu powolnej zmiany klimatu sytuacja ta może ulec zmianie, przez co modele takie mogą dostarczać ważnych informacji o skutkach tak gwałtownych zjawisk atmosferycznych jak np. tornada.